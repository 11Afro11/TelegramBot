\documentclass[0.01pt,a4paper,spanish]{article}

\usepackage[spanish]{babel}
\usepackage[utf8]{inputenc}
\usepackage{amsmath, amsthm}
\usepackage{amsfonts, amssymb, latexsym}
\usepackage{enumerate}
\usepackage[official]{eurosym}
\usepackage{graphicx}
\usepackage[usenames, dvipsnames]{color}
\usepackage{colortbl}
\usepackage{multirow}
\usepackage{fancyhdr}
\usepackage[all]{xy}
\usepackage[below]{placeins}
\usepackage{subfigure}
\usepackage{eso-pic}
\usepackage{lastpage}

\newcommand\BackgroundPic{
\put(0,0){
\parbox[b][\paperheight]{\paperwidth}{%
\vfill
\centering
\includegraphics[width=\paperwidth,height=\paperheight,
keepaspectratio]{hola.jpg}
\vfill
}}}

\DeclareGraphicsExtensions{.png,.pdf,.jpg,.gif }



\title{Examen Teórico de la Constitución Española de 1978}
\author{Nombre:............................  Apellidos:..............................................}
\date{\today}
\begin{document}
\AddToShipoutPicture{\BackgroundPic}
\maketitle
\textbf{Responda a las preguntas marcando una X en la casilla correspondiente.}
\begin{enumerate}
\item - El derecho de peticion:
\begin{itemize} 
\item [$\square$] a) Podra ser ejercito en determinados casos de forma individual.
\item [$\square$] b) No podra ser ejercido colectivamente.
\item [$\square$] c) Esta reconocido en beneficio de los españoles en el art. 30 de la Constitucion.
\item [$\square$] d) Siempre debera ser ejercido por escrito.
\end{itemize}
\item - El control del ejercicio de las funciones delegadas por el Poder Ejecutivo central a las Comunidades Autonomas lo ejercera:
\begin{itemize} 
\item [$\square$] a) El Gobierno central.
\item [$\square$] b) El Tribunal Constitucional.
\item [$\square$] c) El Gobierno central, previo dictamen del Consejo de Estado.
\item [$\square$] d) A la jurisdiccion contencioso administrativa.
\end{itemize}
\item - ¿En que articulo de la Constitucion se recoge el juramento del Principe Heredero?
\begin{itemize} 
\item [$\square$] a) En el articulo 61.3.
\item [$\square$] b) En el articulo 61.2.
\item [$\square$] c) En el articulo 64.2.
\item [$\square$] d) En el articulo 63.1.
\item [$\square$] e) En el articulo 63.5.
\item [$\square$] f) En el articulo 70.1.
\end{itemize}
\item - Solo uno de los siguientes derechos podra ser suspendido en caso de se acuerde la declaracion de estado de excepcion o de sitio:
\begin{itemize} 
\item [$\square$] a) Derecho de huelga.
\item [$\square$] b) Derecho de peticion colectiva.
\item [$\square$] c) Derecho a obtener la tutela efectiva de Jueces y Tribunales.
\item [$\square$] d) Derecho de creacion de centros docentes.
\end{itemize}
\item - Solo uno de los siguientes derechos podra ser suspendido en caso de se acuerde la declaracion de estado de excepcion o de sitio:
\begin{itemize} 
\item [$\square$] a) Derecho a participar en los asuntos publicos.
\item [$\square$] b) Derecho de asociacion.
\item [$\square$] c) Derecho a acceder en condiciones de igualdad a las funciones y cargos publicos.
\item [$\square$] d) Derecho a la negociacion colectiva laboral.
\end{itemize}
\item - El Gobierno esta compuesto por:
\begin{itemize} 
\item [$\square$] a) El Presidente y el Consejo de Estado.
\item [$\square$] b) El Presidente y las Comisiones Delegadas del Gobierno.
\item [$\square$] c) El Presidente, los Vicepresidentes en su caso, los Ministros y los demas miembros que establezca la Ley.
\item [$\square$] d) El Rey, el Presidente y los Ministros.
\end{itemize}
\item - ¿Cuantas Salas tiene la Audiencia Nacional?:
\begin{itemize} 
\item [$\square$] a) Cuatro
\item [$\square$] b) Dos
\item [$\square$] c) Tres
\item [$\square$] d) Cinco
\end{itemize}
\item - La estructura organica de la Presidencia del Gobierno se aprueba por el/la:
\begin{itemize} 
\item [$\square$] a) Ley.
\item [$\square$] b) Consejo de Ministros.
\item [$\square$] c) Presidente del Gobierno.
\item [$\square$] d) Ministro de la Presidencia.
\end{itemize}
\item - ¿A cual de los siguientes cargos no nombra el Rey?
\begin{itemize} 
\item [$\square$] a) Al Fiscal General del Estado.
\item [$\square$] b) Al Presidente del Tribunal Supremo.
\item [$\square$] c) A los vocales de los Tribunales Consuetudinarios.
\item [$\square$] d) A los vocales del Consejo General del Poder Judicial.
\end{itemize}
\item - No corresponde la iniciativa legislativa:
\begin{itemize} 
\item [$\square$] a) Al Congreso.
\item [$\square$] b) Al Senado.
\item [$\square$] c) Al Gobierno.
\item [$\square$] d) A las Cortes Generales.
\end{itemize}
\end{enumerate}
\bigskip
\begin{center}

¡BUENA SUERTE!

\end{center}
\newpage

\ClearShipoutPicture
\end{document}
